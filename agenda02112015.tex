\documentclass{agenda}
	\usepackage{url}
	\group{CS Women General Group}
	\dateproposed{11 Feb. 2015}
	\objective{Delegation, First lunch}
	\invited{CS Women General Membership}
\begin{document}

	\agendaheader{}
	
	\begin{agendaitem}{5}{Co-Chair Introduction}
		Introduce the new co-chairs. Elevator pitch of the goals for this semester.
	\end{agendaitem}
	
	\begin{agendaitem}{10}{Administrivia}
		Report on census results. Gauge interest in reforming the group.
		\begin{itemize}
			\item \actionitem{Sign-up sheet for forum to discuss group goals.}
			\item \actionitem{Sign-up sheet for group to discuss GSO status.}
			\item \actionitem{Sign-up sheet for recruiting dinner.}
			\item \actionitem{Sign-up sheet for }
		\end{itemize}
	\end{agendaitem}
	
	\begin{agendaitem}{20}{Laura Dietz:  Queripidia - Query-specific Wikipedia Construction}
		We all turn towards Wikipedia with questions we want to know more about, but eventually find ourselves on the limit of its coverage. Instead of providing "ten blue links", my goal is to answer any web query with something that looks and feels like Wikipedia. I am developing algorithms to automatically retrieve, extract, and compile a knowledge resource for a given web query. For a very early web demo with some example queries see: \url{http://ciir.cs.umass.edu/~dietz/queripidia/}
	\end{agendaitem}
	
	\begin{agendaitem}{15}{Laura Dietz: Build a shrine to yourself}
	Am I good enough? Is this the right topic? Am I in the right place? Am I passionate enough? If I switch now, will my career be over? - I have a lot of self doubts. I went through several advisors, topics, and research labs before I graduated age 34. Maybe I did everything wrong. Yet, some people believe that I am a successful researcher. How did I trick people? This is a talk about how to build a shrine to yourself.
	\end{agendaitem}
	
	\begin{agendaitem}{10}{Discussion}
	\end{agendaitem}
	
\end{document}
